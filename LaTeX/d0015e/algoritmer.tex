\documentclass[a4paper, 12pt]{article}
\usepackage[utf8]{inputenc}
\usepackage[swedish]{babel}
\usepackage{a4wide}

\title{Algoritmer} % TODO fixa titeln
\author{Gustav Hansson \thanks{email: \texttt{gushan-6@student.ltu.se}}\\
    ~\\
    Institutionen för system- och rymdteknik \\
    Luleå Tekniska Universitet \\
    971 87 Luleå, Sverige}

\date{\today}

\begin{document}

\maketitle

\begin{abstract}
    Blabla asöldfhjlaksdf kh laksdf asdf asdfh lkajsdfkl s %TODO gör sammanfattning
\end{abstract}

\section{Problemet}

    \subsection{"Big-O"}
        Big-O är ett simpelt sätt att beskriva hur effektiv en algoritm är.
        Den skrivs som $O(N^x)$ där $N$ står för mängden data och $x$ står för
        den proportionella antalet körningar. Detta menas med att om $O(N^2)$ så
        kommer tiden ha en ekponentiell ökning. Fördelen med att skriva på detta
        sett är att det är väldigt simpelt att få fram och i många fall krävs inte
        exakta beräkningar.

        Nakdelen med detta sätt att skriva är då självklart att det är väldigt inexakt.
        Det säger egentligen ingenting om hur lång tid programmet kommer kräva då
        exempelvis $O(N^2)$ innefattar både $25N^2 - 10N$ och $\frac{N^2}{2} + 5N$.
        Dessa två funktioner är helt olika men ingår ändå i $O(N^2)$.



\end{document}
